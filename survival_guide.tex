\documentclass{article}
\usepackage[a4paper, total={6in, 11.5in}, margin=1in]{geometry}
\usepackage{graphicx} % Required for inserting images
\usepackage{xcolor}
\usepackage{hyperref}
\usepackage{soul} % For underlining

% Configure hyperref for colored and underlined links
\hypersetup{
    colorlinks=true,
    linkcolor=cyan,
    filecolor=magenta,
    urlcolor=cyan,
    pdftitle={BMES 441/641 Survival Guide},
    pdfpagemode=FullScreen,
}

% Define a command for underlined links
\newcommand{\ulink}[2]{\href{#1}{\ul{#2}}}

% \pagenumbering{gobble}
\title{BMES 441/641 Survival Guide v0.1}
\author{Christian D'Andrea }
\date{September 2024}
\begin{document}
\maketitle

\section{Introduction}
Reports show me that you know how to think logically about the experiment performed. That you can envision an experiment and think of relevant, specific, and testable hypotheses. That you can provide background about why this work is relevant or useful, what is to be learned, and what the objective is to complete this learning. That you can describe your methods such that the process could be reproduced. That your results transparently describe the outcomes without interpretation (yet). That the discussion provides an evidence-backed interpretation of your results with respect to your hypotheses. If you can prove to me that you succeed at this within the required formatting guidelines, you will likely achieve a grade with which you'll be content!

\section{Presentation and Report Submission Tips}
\begin{enumerate}
\item Submit presentations and reports as a PDF directly to BBLearn. Do NOT submit the PDF in a zip file.
\item Code will be submitted either in a zip file or to GitHub (TBD)
\end{enumerate}

\section{Statistics}
Here is a choose-your-own-adventure guide to performing statistics on your data. This should cover anything you run into during these courses. I'm not a statistician, and am learning this art along with you, so if you disagree with any recommendation below, please email me and we can figure it out.

\begin{enumerate}
    \item Pre-tests
    \begin{enumerate}
    \item Normality Test
    \begin{enumerate}
    \item For small samples ($n < 50$): Shapiro-Wilk test
    \item For large samples ($n \geq 50$): Kolmogorov-Smirnov test
    \item Visual inspection (Q-Q plots, histograms)
    \end{enumerate}
    \item Homogeneity of Variance (for parametric tests)
    \begin{enumerate}
    \item Levene's test
    \item Bartlett's test (if data is normal)
    \end{enumerate}
    \item Outlier Detection
    \begin{enumerate}
    \item Visual methods (box plots, scatter plots)
    \item Statistical methods (Z-scores, Interquartile Range)
    \end{enumerate}
    \end{enumerate}
    \item Test Selection Guide
    \begin{enumerate}
        \item Two Groups
        \begin{enumerate}
            \item Small Sample Size ($n < 30$ per group)
            \begin{enumerate}
                \item Normal Distribution \& Equal Variances
                \begin{itemize}
                    \item Independent: Independent samples t-test
                    \item Paired: Paired samples t-test
                \end{itemize}
                \item Non-Normal Distribution OR Unequal Variances
                \begin{itemize}
                    \item Independent: Mann-Whitney U test
                    \item Paired: Wilcoxon signed-rank test
                \end{itemize}
            \end{enumerate}
            \item Large Sample Size ($n \geq 30$ per group)
            \begin{enumerate}
                \item Normal Distribution
                \begin{itemize}
                    \item Independent: Independent samples t-test
                    \item Paired: Paired samples t-test
                \end{itemize}
                \item Non-Normal Distribution
                \begin{itemize}
                    \item Independent: Mann-Whitney U test (or t-test if $n > 50$)
                    \item Paired: Wilcoxon signed-rank test (or paired t-test if $n > 50$)
                \end{itemize}
            \end{enumerate}
        \end{enumerate}
        \item More than Two Groups
        \begin{enumerate}
            \item Small Sample Size ($n < 30$ per group)
            \begin{enumerate}
                \item Normal Distribution \& Equal Variances
                \begin{itemize}
                    \item Independent: One-way ANOVA
                    \item Paired/Repeated Measures: Repeated measures ANOVA
                \end{itemize}
                \item Non-Normal Distribution OR Unequal Variances
                \begin{itemize}
                    \item Independent: Kruskal-Wallis H test
                    \item Paired/Repeated Measures: Friedman test
                \end{itemize}
            \end{enumerate}
            \item Large Sample Size ($n \geq 30$ per group)
            \begin{enumerate}
                \item Normal AND Non-Normal Distributions
                \begin{itemize}
                    \item Independent: One-way ANOVA
                    \item Paired/Repeated Measures: Repeated measures ANOVA
                \end{itemize}
            \end{enumerate}
        \end{enumerate}
    \end{enumerate}

    \item Post-hoc Tests
    \begin{enumerate}
        \item For ANOVA (parametric)
        \begin{itemize}
            \item Tukey's HSD (Honestly Significant Difference): For all pairwise comparisons when sample sizes are equal and variances are homogeneous
            \begin{itemize}
                \item Default in MATLAB's \ulink{https://www.mathworks.com/help/stats/multiple-comparisons.html\#bum7ue_-1}{multcompare()} function
                \item Balances Type I error mitigation and statistical power
                \item Suitable for most situations with equal sample sizes
            \end{itemize}
            \item Bonferroni: Conservative approach for mitigating Type I error, suitable for a small number of comparisons, or else it may get \textit{too} conservative. Consider using this when the real-life consequences of a Type I error are high.
            \item Dunnett's test: When comparing multiple groups to a single control group
            \item This is not an exhaustive list, but likely covers most cases. Feel free to use another if justified.
        \end{itemize}
        \item For Kruskal-Wallis H test (non-parametric)
        \begin{itemize}
            \item Dunn's test: For pairwise comparisons after a significant Kruskal-Wallis result
            \item Mann-Whitney U tests with Bonferroni correction: Alternative to Dunn's test, more conservative
        \end{itemize}
        \item For Friedman test (non-parametric repeated measures)
        \begin{itemize}
            \item Nemenyi test: For all pairwise comparisons
            \item Wilcoxon signed-rank tests with Bonferroni correction: Alternative to Nemenyi test, more conservative
        \end{itemize}
    \end{enumerate}
    
    \item Other Cases
    \begin{enumerate}
        \item Categorical Data
        \begin{itemize}
            \item Two categories: Chi-square test of independence
            \item More than two categories: Chi-square test or Fisher's exact test (for small expected frequencies)
        \end{itemize}
        \item Correlation
        \begin{itemize}
            \item Normal Distribution: Pearson correlation
            \item Non-Normal Distribution: Spearman rank correlation
        \end{itemize}
        \item Regression
        \begin{itemize}
            \item Linear relationship \& Normal residuals: Linear regression
            \item Non-linear relationship or Non-normal residuals: Non-linear regression or Generalized Linear Models / Multiple-linear regression
        \end{itemize}
    \end{enumerate}
\end{enumerate}


\section {Useful Tools}
This class involves a solid amount of coding work, no doubt. The important part is not that you know how to change a plot's line width in MATLAB, it's that you, with your brain, can architect an flow which allows you to perform a valid analysis of the data and intuitively visualize it, and communicate this effectively to me/your readers, in a way that proves you have designed and understand your approach. This approach still requires knowledge of how to write and generate well-structured code, but it's wasted time to get stuck on small details which don't enable learning biomechanics. So, here are some helpful tools to this end, which I use in my research:

\begin{enumerate}
\item \href{https://miro.com/}{Miro}, a flowchart/diagramming tool
\item \href{https://www.cursor.com/}{Cursor}, an AI-enabled code editor. You can install a MATLAB extension for editing the code in Cursor, and can run this code through the usual MATLAB IDE.
\item \href{https://www.github.com}{GitHub}, a Software Versioning and Collaboration tool
\item \href{https://www.perplexity.ai/}{Perplexity}, a new search engine - great for gathering initial evidence, leading more quickly to scholarly sources
\item \href{https://notebooklm.google.com/}{NotebookLM}, an interface to upload resarch and chat with it, as well as generate podcast-like audio summaries of the material. Great for gaining familiarity with a paper without having to read it.
\end{enumerate}

\end{document}
